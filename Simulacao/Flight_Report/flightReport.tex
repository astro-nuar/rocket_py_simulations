\documentclass{article}
% ----------------------------------------------------------------------
% Define external packages, language, margins, fonts, new commands 
% and colors
% ----------------------------------------------------------------------
\usepackage[utf8]{inputenc} % Codification
\usepackage[english]{babel} % Writing idiom

\usepackage{pgfplots}
\pgfplotsset{compat=1.18}
\usepackage[T1]{fontenc}
\usepackage{imakeidx}
\usepackage{circuitikz}
\usetikzlibrary{positioning}
\usepackage[export]{adjustbox} % Align images
\usepackage{amsmath} % Extra commands for math mode
\usepackage{amssymb} % Mathematical symbols
\usepackage{anysize} % Personalize margins
    %\marginsize{2cm}{2cm}{2cm}{2cm} % {left}{right}{above}{below}
\usepackage{appendix} % Appendices
\usepackage{cancel} % Expression cancellation
\usepackage{caption} % Captions
    \captionsetup{labelfont={bf}}
\usepackage{cite} % Citations, like [1 - 3]
\usepackage{color} % Text coloring
\usepackage{fancyhdr} % Head note and footnote
    \pagestyle{fancy}
    \fancyhf{}
    \fancyhead[L]{
        \includegraphics[width=4cm]{Images/NovaFct.png}
    } % Left of Head note
    \fancyhead[R]{\footnotesize Euroc} % Right of Head note
    \fancyfoot[L]{\footnotesize NUAR} % Left of Footnote
    \fancyfoot[R]{\thepage  } % Center of Footnote
    %\fancyfoot[R]{\footnotesize Degree} % Right of Footnote
    \renewcommand{\footrulewidth}{0.4pt} % Footnote rule
\usepackage{float} % Utilization of [H] in figures
\usepackage{graphicx} % Figures in LaTeX
\usepackage[colorlinks = true, plainpages = true, linkcolor = istblue, urlcolor = istblue, citecolor = istblue, anchorcolor = istblue]{hyperref}
\usepackage{indentfirst} % First paragraph
\usepackage[super]{nth} % Superscripts
\usepackage{siunitx} % SI units

\usepackage{tocloft}%style the listofcontent to have dots
\renewcommand{\cftsecleader}{\cftdotfill{\cftdotsep}}

\usepackage[list=true, listformat=simple]{subcaption} % Subfigures

\usepackage{titlesec} % Font
    \titleformat{\section}{\Large\bfseries}{\thesection}{1em}{}
    \titleformat{\subsection}{\large\bfseries}{\thesubsection}{1em}{}
    \titleformat{\subsubsection}{\normalsize\bfseries}{\thesubsubsection}{1em}{}
    %\fancyfoot[C]{\thepage}

%code
\usepackage[breakable]{tcolorbox}

\graphicspath{{Images/.}{../Images/.}}

% Code highlighting
\usepackage{listings}
\lstset{ 
    backgroundcolor=\color{cellbackground}, % background color for the code block
    basicstyle=\ttfamily,                   % font style
    breakatwhitespace=false,                % automatic breaks only at whitespace
    breaklines=true,
    showspaces=false,
    showstringspaces=false,
    showtabs=false,                        % automatic line breaking
    captionpos=b,                           % caption position
    commentstyle=\color{commentcolour},     % comment style
    escapeinside={\%*}{*)},                 % if you want to add LaTeX within your code
    keywordstyle=\color{blue},              % keyword style
    stringstyle=\color{string},             % string style
    frame=single,                           % adds a frame around the code
    rulecolor=\color{cellborder},           % border color
}

%pseudocode styling
\newcounter{nalg}[section] % defines algorithm counter for chapter-level
\DeclareCaptionLabelFormat{simple}{Algoritmo \thenalg} % defines a new caption label as Algorithm x.y
\lstnewenvironment{algorithm}[1][] % defines the algorithm listing environment
{
    \vspace{1ex}
    \refstepcounter{nalg} % increments algorithm number
    \captionsetup{labelformat=simple,labelsep=colon,position=top} % set caption position to top
    \lstset{ % this is the style
        mathescape=true,
        frame=tB,
        numbers=left, 
        numberstyle=\tiny,
        basicstyle=\scriptsize, 
        keywordstyle=\color{black}\bfseries\em,
        keywords={input, output, return, datatype, function, in, if, else, foreach, while, begin, end}, % add the keywords you want, or load a language as Rubens explains in his comment above.
        numbers=left,
        xleftmargin=.04\textwidth,
        #1 % this is to add specific settings to an usage of this environment (for instance, the caption and referable label)
    }
}
{}

% Exact colors from NB
\definecolor{incolor}{HTML}{303F9F}
\definecolor{outcolor}{HTML}{D84315}
\definecolor{cellborder}{HTML}{CFCFCF}
\definecolor{cellbackground}{HTML}{F7F7F7}
\definecolor{commentcolour}{rgb}{0.42, 0.45, 0.51}
\definecolor{string}{rgb}{0.42, 0.45, 0.51}

\usetikzlibrary{shapes,arrows}
\tikzstyle{block} = [draw, rectangle, 
    minimum height=2.5em, minimum width=3.5em]
\tikzstyle{sum} = [draw, circle, node distance=1cm]
\tikzstyle{input} = [coordinate]
\tikzstyle{output} = [coordinate]
\tikzstyle{pinstyle} = [pin edge={to-,thin,black}]


% Random text (not needed)
\usepackage{lipsum}
\usepackage{duckuments}

% New and re-newcommands
\newcommand{\sen}{\operatorname{\sen}} % Sine function definition
\newcommand{\HRule}{\rule{\linewidth}{0.5mm}} % Specific rule definition
\renewcommand{\appendixpagename}{\LARGE Appendices}

% Colors
\definecolor{istblue}{RGB}{28, 118, 196}
\definecolor{dkgreen}{rgb}{0,0.6,0}

%%%%%%%%%%%%%%%%%%%%%%%%%%%%%%%%%%%%%%%%%%%%%%%%%%%%%%%%%%%%%%%%%%%%%%%%
%                                 Document                             %
%%%%%%%%%%%%%%%%%%%%%%%%%%%%%%%%%%%%%%%%%%%%%%%%%%%%%%%%%%%%%%%%%%%%%%%%
\begin{document}

% ----------------------------------------------------------------------
% Cover
% ----------------------------------------------------------------------
\begin{center}
    \begin{figure}
        \vspace{-1.0cm}
        \includegraphics[scale = 1, left]{Images/NovaFct.png} % Nova logo
    \end{figure}

    \mbox{}\\[2.0cm]
    \textsc{\Huge Núleo de Aero Espacial}\\[2.5cm]
    \textsc{\LARGE Euroc}\\[2.0cm]
    \HRule\\[0.4cm]
    {\large \bf { 
        Simulation Results Report\\
    }[\texttt{EN}]}\\[0.2cm]
    \HRule\\[1.5cm]
\end{center}

\begin{flushleft}
    \textbf{Authors:}
\end{flushleft}

\begin{center}
    \begin{minipage}{0.5\textwidth}
        \begin{flushleft}
            Filipe Cavalheiro (62894)
        \end{flushleft}
    \end{minipage}%
    \begin{minipage}{0.5\textwidth}
        \begin{flushright}
            \href{mailto:fs.cavalheiro@campus.fct.unl.pt}{\texttt{fs.cavalheiro@campus.fct.unl.pt}}
        \end{flushright}
    \end{minipage}
\end{center}

\vspace{4cm}

\begin{center}
    \large \bf 2023/2024 -- 2º Semester
\end{center}

\thispagestyle{empty}

\setcounter{page}{0}

\newpage

\newpage

\tableofcontents % Generates the table of contents

\newpage

\listoffigures

\newpage
\section{Introduction}

This reports serves to show all important data collected by the simulations

\section{Obtained Results}
\label{sec:resultados}
The plots made by Rocket py go as follows:

\begin{figure}[H]
    \centering
    \includegraphics[width=\textwidth]{trajectory_3d_plot.png} 
    \label{fig:aerodynamic_forces}
\end{figure}

\begin{figure}[H]
    \centering
    \includegraphics[width=\textwidth]{stability_and_control_data.png} 
    \label{fig:aerodynamic_forces}
\end{figure}

\begin{figure}[H]
    \centering
    \includegraphics[width=\textwidth]{rail_buttons_forces.png} 
    \label{fig:aerodynamic_forces}
\end{figure}

\begin{figure}[H]
    \centering
    \includegraphics[width=\textwidth]{pressure_rocket_altitude.png} 
    \label{fig:aerodynamic_forces}
\end{figure}

\begin{figure}[H]
    \centering
    \includegraphics[width=\textwidth]{linear_kinematics_data.png} 
    \label{fig:aerodynamic_forces}
\end{figure}

\begin{figure}[H]
    \centering
    \includegraphics[width=\textwidth]{fluid_mechanics_data.png} 
    \label{fig:aerodynamic_forces}
\end{figure}

\begin{figure}[H]
    \centering
    \includegraphics[width=\textwidth]{flight_path_angle_data.png} 
    \label{fig:aerodynamic_forces}
\end{figure}

\begin{figure}[H]
    \centering
    \includegraphics[width=\textwidth]{energy_data.png} 
    \label{fig:aerodynamic_forces}
\end{figure}

\begin{figure}[H]
    \centering
    \includegraphics[width=\textwidth]{attitude_data.png} 
    \label{fig:aerodynamic_forces}
\end{figure}

\begin{figure}[H]
    \centering
    \includegraphics[width=\textwidth]{angular_kinematics_data.png} 
    \label{fig:aerodynamic_forces}
\end{figure}

\begin{figure}[H]
    \centering
    \includegraphics[width=\textwidth]{aerodynamic_forces.png} 
    \label{fig:aerodynamic_forces}
\end{figure}

\section{Conclusions}
I conclude that i dont know enought

\pagebreak
\begin{thebibliography}{9}
  
\end{thebibliography}

\end{document}